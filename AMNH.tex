\documentclass[12pt]{article}
%\documentclass[12pt]{exam}
\usepackage[includeheadfoot, top=1in, bottom=1in, hmargin=1in]{geometry}
\usepackage{fancyhdr}
\usepackage{verbatim}
\usepackage{url}
\pagestyle{fancy}
\usepackage[latin1]{inputenc}
\usepackage{amsmath}
\usepackage[pdftex]{graphicx}
\usepackage[english]{babel}
\usepackage{amsfonts}
\usepackage{amssymb}
\usepackage{setspace}
\usepackage{}

\newcommand{\degrees}{\ensuremath{^\circ}}
\newcommand{\arcmin}{\ensuremath{'}}
\newcommand{\arcsec}{\ensuremath{"}}
\newcommand{\hours}{\ensuremath{^\mathrm{h}}}
\newcommand{\minutes}{\ensuremath{^\mathrm{m}}}
\newcommand{\seconds}{\ensuremath{^\mathrm{s}}}

\newcommand{\s}[0]{\phantom{i}} %sets up \s command
\newcommand{\m}[0]{\phantom{abcde}} %sets up \m command
\providecommand{\e}[1]{\ensuremath{\times 10^{#1}}} %sets up \e command

\graphicspath{{./}{figures/}}
%\doublespacing
\singlespacing

\chead{}
\rhead{Astronomy Lab} \lhead{Spring 2023}
\renewcommand{\rightmark}{}

\begin{document}

\setlength{\parskip}{8pt plus2pt minus2pt}

\begin{center}

\Large\textbf{AMNH Rose Center for Earth and Space}
%\large{Visit 3/10/23}
\end{center}

\vspace{5 pt}

\section{Worlds Beyond Earth (Post-show questions)}

What did you learn about the early solar system from space show?  Describe something about the evolution of the solar system that you didn't know before.

\vspace{2.25in}


\section{Weight on astronomical objects}
Strewn throughout the Rose Center (primarily on the bottom floor) are several different scales on which you can stand.  These scales tell you how much you might weigh, were you to stand on the surface of different astronomical bodies (e.g. the Moon, Halley's Comet, a neutron star, etc.)  

\begin{enumerate}
\item Stand on several different scales.  Write down how much you weigh on each and the name of the corresponding astronomical body.  (If the scales aren't working, you can use \url{https://www.exploratorium.edu/explore/solar-system/weight} to calculate your weight on different astronomical bodies.)

\begin{enumerate}
\item
%\vspace{.2in}

\item
\vspace{.2in}

\item
\vspace{.2in}
\end{enumerate}
%\vspace{.2in}

\item \emph{Weight} is not a term often found in physics because it is a special case of a more universal quantity we use.  What is the name of this more universal
quantity?  (Hint -- it is one of these: \emph{acceleration, work, power, energy, mass, force, velocity, momentum}.)  How does it relate to weight specifically?
%Think about this one a bit.  What does a scale \emph{actually} measure?  How do you think it works?  What are ways you can change its reading (even momentarily)?  Explain.
\vspace{1.25in}

\item On which object did you weigh the most?  On which object did you weigh the least?  
\vspace{0.5in}

\item Using your results from the previous questions, why would you have a different weight on each astronomical body?  Why wouldn't you just weigh what you do on Earth?  What is different about 
these bodies that is responsible for changing your weight on them?  (Hint: two 
factors!)
\vspace{1.75in}

%\item Can you think of an equation that you have learned in your courses or lab that relates these two factors to the quantity from question 2?  You \emph{have} covered this equation, earlier this semester.  
%\vspace{1in}

%\item These scales don't actually teleport you to the surfaces of other astronomical objects, so how do you think these work to approximate what you would weigh in these exotic locales? 
%\vspace{1in}
\end{enumerate}


\section{Meteorites}

There's a lot more to the Solar System than just planets! In the Hall of
Meteorites at the Museum of Natural History, we can learn about the other solid
bodies that orbit the Sun. Although smaller, they have played a role in the
evolution of life on Earth, and they carry records that tell the history of the
Solar System going all the way back to the formation of the planets (and even
further!).  


\emph{After} taking time to explore all of the exhibits in the 
Hall of Meteorites, answer the questions below in your lab notebook. Some answers require 
you to combine information from more than one exhibit. 


\subsection{The Ahnighito Meteorite}

\begin{enumerate}
\item What makes this meteorite so special that it's the centerpiece of the
collection at AMNH?

\vspace{1in}

%\newpage
\item What is the size of the Ahnighito meteorite relative to a typical human adult?  Make a rough sketch of the meteorite's outline along with a person (any of your classmates will do) standing next to it. 



\vspace{3.5in}
%\newpage

\item What is the meteorite mostly made of?  How heavy is the Ahnighito meteorite?
If you assume that an average person weighs 150 lbs, 
how many people would it take to balance out the weight of this relatively compact 
meteorite?  (Note: 1 ton = 2000 lbs)

\vspace{1.5in}

\item What is the difference between a meteorite and a meteor? 

\vspace{1.5cm}

\item Give an example of a good place on Earth to search for meteorites.  What makes this type of location optimal?

\vspace{2.5cm}


\subsection{Impacts}

\item Why is the Moon's face covered in craters, while we only see a few on Earth? 
%Explain.

\vspace{2.5cm}

\item How does the impact of a large body lead to a mass extinction event, as in the
case of the dinosaurs?  (Hint: see the Willamette Meteorite display on the lowest level of the Rose Center.)

\vspace{3cm}

\item When and where do scientists think Chicxulub (the dinosaur-killing impact) 
occurred?  What is the evidence for it being responsible for this mass
extinction?

\vspace{3cm}


\item Briefly explain the leading theory of how the Moon formed. 

\vspace{2.5cm}

%\newpage
%\subsection{The Early Solar System}

%\item If you were to take a small amount of the solar surface and cool it to
%room temperature, which objects would it resemble in composition?  Why?

%\vspace{1cm}

\item What do you think are the main reasons scientists want to study meteors/meteorites?  %Please cite two reasons.

\end{enumerate}
\vspace{2cm}




\section{The Big Bang}
At the end of the \emph{Scales of the Universe} exhibit, you should find yourself at the entrance to the big sphere.  You may have to wait a minute or two in line to get inside to see a short movie on the Big Bang.  See this short film, then answer these questions.

\begin{enumerate}
\item How old is the Universe?
\vspace{0.3in}

\item What happened as the Universe expanded?
\vspace{1in}

\item How did small galaxies interact with each other as the Universe evolved?
\vspace{1.5in}

\end{enumerate}


\section{Black Hole Movie}

The AMNH has a small theater on the lowest floor under the sphere and in the corner near the images of the galaxies on the wall.  It cycles through two or three films, each lasting 5-10 minutes.  Go watch the black hole film. (You can watch the other ones too -- they are interesting!)  After watching it, answer these questions:
%The Black Hole exhibit in the Cullman Hall of the Universe

\begin{enumerate}

%\item Why do black holes act as gravitational lenses?
%\vspace{1in}

\item How do black holes form?
\vspace{0.7in}

\item If black holes are, in fact, ``black", how do astrophysicists detect them?
\vspace{1in}
\end{enumerate}


\section{Conclusions}
%As a member of this lab, your knowledge of the physical sciences (in particular, the space sciences) is greater than the layperson.  As such, I would like you to critique one of the astronomy-related exhibits where you can apply your advanced education.  %You can pick any of the exhibits not directly used in the preceding questions in this lab.  However, you must choose an exhibit that you have in some way covered in previous labs/class this year.  
%Possible examples are the spectra exhibit, galaxy evolution exhibit, stellar evolution and the HR diagram, comets, one of the other astronomy films, etc.

What was your favorite part of visiting The Rose Center for Earth and Space and why? What is something new that you learned?


\end{document}